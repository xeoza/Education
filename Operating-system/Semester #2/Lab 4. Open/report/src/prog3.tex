% !TEX root = main.tex
\section{testFOpen.c}

\biglisting{../../testFOpen.c}

\paragraph{Вывод}\hfill\\

\biglisting{../../testFOpen_output.txt}


\paragraph{Анализ}\hfill\\

С помощью функций $fopen()$ открываем два потока на запись с начала файла. Они имеют два разных файловых дескриптора и следовательно независимые позиции в файле. Затем в цикле с помощью $fprintf()$ в потоки поочередно записываются буквы от a до z, т.е. нечетные в первый поток, четные во второй. Следует помнить о том, что функция $fprintf()$ обеспечивает буферизацию. Запись непосредственно в сам файл происходит в трех случаях: либо при полном заполнении буфера, либо при вызове функций $fclose()$ и $fflush()$. Функция $fclose()$ отделяет указанный поток от связанного с ним файла или набора функций. Если поток использовался для вывода данных, то все данные, содержащиеся в буфере, сначала записываются с помощью $fflush()$. Функция $fflush()$ принудительно записывает все буферизированные данные в устройство вывода данных. При этом поток остается открытым. Т.к. оба потока открыты в режиме перезаписи в файл, то после выполнения второго $fclose()$ данные в файле, записанные с помощью первого потока, будут переписаны и мы увидим там буквы, чтоящие на нечетных местах.


