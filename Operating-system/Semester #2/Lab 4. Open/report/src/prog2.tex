% !TEX root = main.tex
\section{testKernelIO.c}

\paragraph{Текст программы}\hfill\\

\biglisting{../../testKernelIO.c}

\paragraph{Вывод}\hfill\\

\biglisting{../../testKernelIO_output.txt}

\paragraph{Анализ}\hfill\\

В отличии от первой программы, создаются два файловых дескриптора и две разные записи в системной таблице открытых файлов. Файловые дескрипторы независимы друг от друга, поэтому положения указателей в файле, связанных с данным файловым дескриптором, будут независимы. В результате прохода цикла, где с помощью системных вызовов $read, write$ в $stdout$ выводится по одному символу из файла, увидим строку с дублирующимися буквами.