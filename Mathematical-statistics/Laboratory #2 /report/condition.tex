% !TEX root = main.tex

\section{Постановка задачи}

\paragraph{Цель работы:} построение доверительных интервалов для математического ожидания и дисперсии нормальной случайной величины.

\paragraph{Содержание работы:}

\begin{enumerate}
    \item Для выборки объема $n$ из нормальной генеральной совокупности $X$ реализовать в виде программы на ЭВМ
    \begin{itemize}
        \item вычисление точечных оценок $\hat{\mu}(\vec{X_n})$ и $S^2(\vec{X_n})$ математического ожидания $MX$ и дисперсии $DX$ соответственно;
        \item вычисление нижней и верхней границ $\overline{\mu}(\vec{X_n})$, $\underline{\mu}(\vec{X_n})$ для $\gamma$-доверительного интервала для математического ожидания $MX$;
        \item вычисление нижней и верхней границ $\overline{\sigma^2}(\vec{X_n})$, $\underline{\sigma^2}(\vec{X_n})$ для $\gamma$-доверительного интервала для дисперсии $DX$;
    \end{itemize}
    \item вычислить $\hat{\mu}$ и $S^2$ для выборки из индивидуального варианта;
    \item для заданного пользователем уровня доверия $\gamma$ и $N$ – объема выборки из индивидуального варианта:
    \begin{itemize}
        \item на координатной плоскости $Oyn$ построить прямую $y = \hat{\mu}(\vec{x_N})$, также графики функций $y = \hat{\mu}(\vec{x_n})$, $y = \overline{\mu}(\vec{x_n})$ и $y = \underline{\mu}(\vec{x_n})$ как функций объема $n$ выборки, где $n$ изменяется от 1 до $N$;
        \item на другой координатной плоскости $Ozn$ построить прямую $z = S^2(\vec{x_N})$, также графики функций $z = S^2(\vec{x_n})$, $z = \overline{\sigma^2}(\vec{x_n})$ и $z = \underline{\sigma^2}(\vec{x_n})$ как функций объема $n$ выборки, где $n$ изменяется от 1 до $N$.
    \end{itemize}
\end{enumerate}