% !TEX root = main.tex

\section{Постановка задачи}

\paragraph{Цель работы:}аппроксимация неизвестной зависимости параболой.

\paragraph{Содержание работы:}
\begin{enumerate}
    \item Для выборки $(y_i, t_i)$, $i = \overline{1; n}$, реализовать в виде программы на ЭВМ:
    \begin{enumerate}
        \item вычисление МНК-оценки $\vec\theta = (\theta_0, \theta_1, \theta_2)$ параметров модели $y = \theta_0 + \theta_1 t + \theta_2 t^2$;
        \item вычисление среднеквадратичного отклонения $\Delta = \sqrt{\sum_{i=1}^{n}\left(y_i - y(t_i)\right)^2}$ полученной модели от результатов наблюдений;
        \item построение на одном графике системы точек $(y_i, t_i)$, $i = \overline{1; n}$, и графика функции $y = y(t)$, $t \in \left[t_{(1)}; t_{(n)}\right]$ (для полученной оценки $\vec\theta$\,).
    \end{enumerate}
    \item провести необходимые вычисления и построить соответствующие графики для выборки из индивидуального варианта.
\end{enumerate}

\paragraph{Содержание отчёта:}
\begin{enumerate}
    \item постановка задачи аппроксимации неизвестной зависимости по результатам наблюдений;
    \item понятие МНК-оценки параметров линейной модели;
    \item формулы для вычисления МНК-оценки в рассматриваемом случае;
    \item текст программы;
    \item результаты расчетов и графики для выборки из индивидуального варианта.
\end{enumerate}