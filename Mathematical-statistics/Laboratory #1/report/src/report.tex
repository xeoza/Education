% !TEX root = main.tex

\section{Отчёт}

\subsection{Формулы для вычисления величин}

\paragraph{Количество интервалов}
\begin{equation}
m = [\log_2 n] + 2
\end{equation}


\paragraph{Минимальное значение выборки}

\begin{equation}
    M_{\min} = \min \{ x_1, \dots, x_n\}, \quad \text{где}
\end{equation}
\begin{itemize}
    \item $(x_1, \dots, x_n)$ --- реализация случайной выборки.
\end{itemize}


\paragraph{Максимальное значение выборки}

\begin{equation}
    M_{\max} = \max \{ x_1, \dots, x_n\}, \quad \text{где}
\end{equation}
\begin{itemize}
    \item $(x_1, \dots, x_n)$ --- реализация случайной выборки.
\end{itemize}


\paragraph{Размах выборки}

\begin{equation}
    R = M_{\max} - M_{\min}, \quad \text{где}
\end{equation}
\begin{itemize}
    \item $M_{\max}$ --- максимальное значение выборки;
    \item $M_{\min}$ --- минимальное значение выборки.
\end{itemize}


\paragraph{Оценка математического ожидания}

\begin{equation}
    \hat{\mu}(\vec{X}_n) = \overline{X}_n = \frac{1}{n} \sum_{i = 1}^{n} X_i\,.
\end{equation}


\paragraph{Выборочная дисперсия}

\begin{equation}
    \hat{\sigma}^2(\vec{X}_n) = \frac{1}{n} \sum_{i = 1}^{n}(X_i - \overline{X}_n)^2 = \frac{1}{n} \sum_{i = 1}^{n}X_i^2 - \overline{X}_n^2\,.
\end{equation}


\paragraph{Несмещённая оценка дисперсии}

\begin{equation}
    S^2(\vec{X}_n) = \frac{n}{n - 1}\hat{\sigma}^2 = \frac{1}{n - 1}\sum_{i = 1}^{n} (X_i - \overline{X}_n)^2\,.
\end{equation}



\subsection{Эмпирическая плотность и гистограмма}

\begin{defn}
    \emph{Эмпирической плотностью распределения случайной выборки $\vec{X}_n$} называют функцию
    \begin{equation}
        f_n(x) =
        \begin{cases}
            \frac{n_i}{n \, \Delta}, &x \in J_i,\; i = \overline{1, m};\\
            0, &\text{иначе}.
        \end{cases}, \quad \text{где}
    \end{equation}
    \begin{itemize}
        \item $J_i,\, i = \overline{1; m}$, --- полуинтервал из $J = [x_{(1)}, x_{(n)}]$, где 
        \begin{align}
            &x_{(1)} = \min\{ x_1, \dots, x_n \}, &x_{(n)} = \max\{ x_1, \dots, x_n \};
        \end{align}
        при этом все полуинтервалы, кроме последнего, не содержат правую границу т.\,е.
        \begin{align}
            &J_i = [ x_{(1)} + (i-1)\Delta, x_{(i)} + i\Delta), \quad i = \overline{1, m-1};
            \\
            &J_m = [ x_{(1)} + (m-1)\Delta, x_{(1)} + m\Delta);
        \end{align}
        \item $m$ --- количество полуинтервалов интервала $J = [x_{(1)}, x_{(n)}]$;
        \item $\Delta$ --- длина полуинтервала $J_i$, $i = \overline{1, m}$ равная
        \begin{equation}
            \Delta = \frac{x_{(n)} - x_{(1)}}{m} = \frac{|J|}{m};
        \end{equation}
        \item $n_i$ --- количество элементов выборки в полуинтервале $J_i$, $i = \overline{1, m}$;
        \item $n$ --- количество элементов в выборке.

    \end{itemize}
\end{defn}

\begin{defn}
    График функции $f_n(x)$ называют гистограммой.
\end{defn}

\subsection{Эмпирическая функция распределения}

\begin{itemize}
    \item $\vec{X}_n = (X_1, \ldots, X_n)$ --- случайная выборка;
    \item $\vec{x}_n = (x_1, \dots, x_n)$ --- реализация случайной выборки $\vec{X}_n$;
    \item $n(x, \vec{x}_n)$ --- количество элементов выборки $\vec{x}_n$, которые меньше $x$.
\end{itemize}

\begin{defn}
    \emph{Эмпирической функцией распределения} называют функцию
    \begin{equation}
        F_n\colon \Re \to \Re, \quad F_n(x) = \frac{n(x, \vec{x}_n)}{n}
    \end{equation}
\end{defn}
\begin{rem}
    $F_n(x)$ обладает всеми свойствами функции распределения. При этом она кусочно-постоянна и принимает значения 
    \[
        0, \frac{1}{n}, \frac{2}{n}, \ldots, \frac{(n-1)}{n}, 1
    \]
\end{rem}
\begin{rem}
    Если все элементы вектора $\vec{x}_n$ различны, то
    \begin{equation}
        F_n(x) = 
        \begin{cases}
            0, & x \leq x_{(1)}; \\
            \frac{i}{n}, & x_{(i)} < x \leq x_{(i+1)},\; i = \overline{1, n-1}; \\
            1, & x > x_{(n)}.
        \end{cases}
    \end{equation}
\end{rem}
\begin{rem}
    Эмпирическая функция распределения позволяет интерпретировать выборку $\vec{x}_n$ как реализацию дискретной случайной величины $\widetilde{X}$ ряд распределения которой
    \begin{center}
        \renewcommand{\arraystretch}{1.5}
        \begin{tabular}{| c || c | c | c |}
            \hline
            $\widetilde{X}$ & $x_{(1)}$ & \ldots & $x_{(n)}$ \\
            \hline
            $\Prob$ & $1/n$ & \ldots & $1/n$ \\
            \hline
        \end{tabular}
    \end{center}
    Это позволяет рассматривать числовые характеристики случайной величины $\widetilde{X}$ как приближённые значения числовых характеристик случайной величины $X$.
\end{rem}